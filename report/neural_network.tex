\Paragraph{Neural Network}
Since the feature contains the sentence structure, the model is designed to process having capability of recognize which word come first and later instead up allowing model randomly search the best combination. Causal convolution layer \ref{causal}, a type of convolution, has been used in this problem to maintain this time series like structure. In order to having a larger size of reception field, dilation blocks are used, in which each layer extract certain amount of feature with sequence information \ref{dilation}, so that it is quite possible to process a much longer sentence without changing the structure with additional information. Residual network structure \ref{res} has been used on dilation block to prevent gradient vanishing in deep network. \\
The overall model structure is a forward causal layer to process sentence in a forward sequence, followed by two dilation residual blocks. There is a parallel structure to process the sentence backward. Two dilation blocks skip connections on both sequence are concatenated, then into a fully connected layer and enters a Softmax layer to make classification prediction.

causal
van den Oord, Aaron, Kalchbrenner, Nal, and Kavukcuoglu, Koray. Pixel recurrent neural networks. ¨
arXiv preprint arXiv:1601.06759, 2016a.

dilation
Aaron van den Oord, Sander Dieleman, Heiga Zen, Karen Simonyan, Oriol Vinyals, Alex
Graves, Nal Kalchbrenner, Andrew Senior, and Koray Kavukcuoglu. Wavenet: A generative
model for raw audio. arXiv preprint arXiv:1609.03499, 2016.

res
He, Kaiming, Zhang, Xiangyu, Ren, Shaoqing, and Sun, Jian. Deep residual learning for image
recognition. CoRR, abs/1512.03385, 2015.