\section{Problem Formulation}

We formualte the word sense disambiguation task into a supervised multi-class
classification problem.
This problem can be represented as tuple $(X,Y,h)$, where $X$ is the instance
space, that contains the description of the object on which we are to predict, 
and $Y$ is the label space, from which each object draws a label,
and $h$ is the classification function $X \rightarrow Y$, that maps the
descriptoin of an object in instance space $X$ to label space $Y$.

Each targetted ambiguous word $w$ has it's own label space $Y_w$, and a shared
instance space $X$ with other ambiguous words.
Therefore, the solution to a word sense disambiguation is specific to each
individual ambiguous word.
For each ambigous word we train a specific classifier $h_w: X
\rightarrow Y_w$, that maps instance from $X$ to it's own label space $Y_w$.


\subsection{Data Preprocessing}
\label{sec:formulate:preprocess}

We use SemCor Corpus~\cite{semcor} to provide the raw data for human languages.
SemCor Corpus is a corpus where each word is tagged with a sense provided by
WordNet~\cite{wordnet}.
We then generate the dataset for this problem, in the following steps:

\begin{enumerate}
  \item Tokenization: transform the sentense strings into list of tokens
    (usually words).
  \item Part-of-Speech tagging: e.g. ``the/DT bar/NN was/VBD crowded/JJ'', where
    DT stands for Determiners, NN stands for Noun, etc.
  \item Lemmatization: e.g. $plays \rightarrow play$, $was \rightarrow be$.
  \item Feature extraction: select some features to represent the context.
\end{enumerate}

\Paragraph{Ambigous word}. 
Given an English word $w$, and a tagged corpus, we say the $w$ is ambiguous, if
it is tagged with at least two different WordNet Sense IDs in the corpus.

\Paragraph{Context Representation}
We use Colocation~\cite{colocation} method to represent the context of a word.
For a given ambiguous word, we looks at the the preceding and succeeding words,
and their Part-of-Speech tags.
The size of the moving window can vary, we use two in this project, as suggested
by previous studies (CITE).
\begin{equation}
  [w_{i-2},POS_{i-2},w_{i-1},POS_{i-1},w_{i+1},POS_{i+1},w_{i+1},POS_{i+1}]
\end{equation}

Vector reprensetations.
Word2Vec(CITE)

\Paragraph{Construct training set for each target word}.
Let W be the set of English words that we want to do sense disambiguation. For
each $w \in W$, we need to extract a training set $X_w$ from the SemCor Corpus
as follow: 
\begin{enumerate}
  \item Draw a set of sentences $S_w$ from SemCor Corpus, where each 
$s \in S_w$ contains w. 
  \item Convert each sentence $s \in S_w$ to a feature vector x, and the sense
    label y. The set of all (x,y) pairs is our training set $X_w$.
\end{enumerate}

\subsection{Performance Measures}. 

We use F1 scores to account for the class imbalance effect in the dataset.
